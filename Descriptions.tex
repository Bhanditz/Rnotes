\documentclass{article}
\usepackage{natbib}

\oddsidemargin  0.1in
\textheight 9.4 in
\textwidth   6.0 in
\topmargin   -0.7in

\renewcommand{\floatpagefraction}{0.8}
\renewcommand{\topfraction}{0.9}
\renewcommand{\bottomfraction}{0.9}

\newcommand{\strutt}{\vrule height 2.5ex depth 0.5ex width 0ex}%> in tabular

\newcommand{\code}[1]{\texttt{#1}}

\title{\Huge Some Example Datasets}

\author{
  {\Large Bill Venables}\\[1.8ex]
  \emph{Department of Statistics}\\
  \emph{The University of Adelaide}
  \\[5mm]
  \copyright\ W. Venables, 1990, 1992.}

\date{}

\begin{document}
\maketitle

\noindent
These datasets and their descriptions are extracted from 
\begin{quotation}
Notes on S: A Programming Environment for Data Analysis and Graphics\\
by Bill Venables (1990)
\end{quotation}

\section{The cloud point data}
\begin{tabbing}
\textbf{Category:} \= \kill
\textbf{Source:} \> Draper \& Smith, \textit{Applied Regression Analysis}, p.\ 162\\
\textbf{Category:} \> Polynomial regression. Simple plots.
\end{tabbing}

\subsection*{Description}
The cloud point of a liquid is a measure of the degree of crystalization
in a stock that can be measured by the refractive index.  It has been
suggested that the percentage of $I_8$ in the base stock is an excellent
predictor of cloud point using the second or third order model:
\[
Y = \beta_0 + \beta_1x + \beta_2x^2 + \beta_3x^3 + E,\qquad E\sim
\mathrm{N}(0,\sigma^2)
\]

\subsection*{Data}
The following data was collected on stocks with known percentage of $I_8$:
\begin{center}
\def\0{\hphantom{0}}
\begin{tabular}{@{\protect\strutt}|cc|cc|cc|cc|}
\hline
$I_8$\% & Cloud Point & $I_8$\% & Cloud Point & $I_8$\% & Cloud Point &
$I_8$\% & Cloud Point \\
\hline
 \00 & 21.9 & \02 & 26.1 & \05 & 28.9 & \08 & 31.4 \\
 \00 & 22.1 & \03 & 26.8 & \06 & 29.8 & \08 & 31.5\\
 \00 & 22.8 & \03 & 27.3 & \06 & 30.0 & \09 & 31.8\\
 \01 & 24.5 & \04 & 28.2 & \06 & 30.3 & 10 & 33.1\\
 \02 & 26.0 & \04 & 28.5 & \07 & 30.4 & &\\
\hline
\end{tabular}
\end{center}

The data may be read from file \code{cloud.data} in a form suitable to
construct a data frame.

\subsection*{Suggested analysis}
Fit polynomial regression models using the \code{lm()} function, and choose
the degree carefully.

\clearpage\section{The Janka hardness data}
\begin{tabbing}
\textbf{Category:} \= \kill
\textbf{Source:} \> E. J. Williams: \textit{Regression Analysis}, Wiley, 1959.\\
\textbf{Category:} \> Polynomial regression.  Transformations.
\end{tabbing}

\subsection*{Description}
The Janka hardness is an important structural property of Australian
timbers, which is difficult to measure.  It is, however, related to the
density of the timber, which is relatively easy to measure.  A low degree
polynomial regression is suggested as appropriate.
\[
Y = \beta_0+\beta_1x+\beta_2x^2+\cdots+E
\]
where $Y$ is the hardness and $x$ the density.

\subsection*{Data}
The following data comes from samples of 36 Australian Eucalypt hardwoods.
\begin{center}
\def\0{\hphantom{0}}
\begin{tabular}{@{\protect\strutt}|cc|cc|cc|cc|cc|cc|}
\hline
D & H & D & H & D & H & D & H & D & H & D & H \\
\hline
 24.7 & 484 & 30.3 &\0587 & 39.4 & 1210 & 42.9 & 1270 & 53.4 & 1880 & 59.8 & 1940\\
 24.8 & 427 & 32.7 &\0704 & 39.9 &\0989 & 45.8 & 1180 & 56.0 & 1980 & 66.0 & 3260\\
 27.3 & 413 & 35.6 &\0979 & 40.3 & 1160 & 46.9 & 1400 & 56.5 & 1820 & 67.4 & 2700\\
 28.4 & 517 & 38.5 &\0914 & 40.6 & 1010 & 48.2 & 1760 & 57.3 & 2020 & 68.8 & 2890\\
 28.4 & 549 & 38.8 & 1070 & 40.7 & 1100 & 51.5 & 1710 & 57.6 & 1980 & 69.1 & 2740\\
 29.0 & 648 & 39.3 & 1020 & 40.7 & 1130 & 51.5 & 2010 & 59.2 & 2310 & 69.1 & 3140\\
\hline
\end{tabular}
\end{center}

The data may be read as a data frame from file \code{janka.data}.

\subsection*{Suggested analysis}
Fit polynomial regression models, choosing the degree carefully.  Examine
the residuals and see if the data has any obvious outliers or
heteroscedasticity.  Check to see what effect a square root or log
transformation has on the residual pattern when plotted against fitted
values.

More advanced: Consider a quasi-likelihood model with variance proportional
to the mean and a square root link.

\clearpage\section{The Tuggeranong house price data}
\begin{tabbing}
\textbf{Category:} \= \kill
\textbf{Source:} \> Dr~Ray Correll, Personal communication\\
\textbf{Category:} \> Multiple regression, coplots.
\end{tabbing}

\subsection*{Description}
Before buying a house in Tuggeranong in February, 1987, a cautious
potential householder collected some data on houses on the market.  The
data for 20 such houses is shown in the table and is available as the
file  \code{house.dat}.  The variables collected are, in order, price,
total floor area, block area, number of main rooms, age of house and
whether or not the house was centrally heated.


\subsection*{Data}
The data is given in Table \ref{tugg} and is available as the file
\texttt{house.data}.

\begin{table}[ht]
\begin{center}
\def\0{\hphantom{0}}
\begin{tabular}{@{\protect\strutt}|*{6}{c|}}
\hline
Price (\$000s) & Floor  (m$^2$) & Block  (m$^2$) & Rooms & Age (years) &
Cent.\ Heat. \\
\hline
52.00 &  111.0  & \0830 &   5   &   6.2 &    no \\
54.75 &  128.0  & \0710 &   5   &   7.5 &    no \\
57.50 &  101.0  &  1000 &   5   &   4.2 &    no \\
57.50 &  131.0  & \0690 &   6   &   8.8 &    no \\
59.75 & \093.0  & \0900 &   5   &   1.9 &   yes \\
62.50 &  112.0  & \0640 &   6   &   5.2 &    no \\
64.75 &  137.6  & \0700 &   6   &   6.6 &   yes \\
67.25 &  148.5  & \0740 &   6   &   2.3 &    no \\
67.50 &  113.5  & \0660 &   6   &   6.1 &    no \\
69.75 &  152.0  & \0645 &   7   &   9.2 &    no \\
70.00 &  121.5  & \0730 &   5   &   4.3 &   yes \\
75.50 &  141.0  & \0730 &   7   &   4.3 &    no \\
77.50 &  124.0  & \0670 &   6   &   1.0 &   yes \\
77.50 &  153.5  & \0795 &   7   &   7.0 &   yes \\
81.25 &  149.0  & \0900 &   6   &   3.6 &   yes \\
82.50 &  135.0  & \0810 &   6   &   1.7 &   yes \\
86.25 &  162.0  & \0930 &   6   &   1.2 &   yes \\
87.50 &  145.0  & \0825 &   6   &   0.0 &   yes \\
88.00 &  172.0  & \0950 &   7   &   2.3 &   yes \\
92.00 &  170.5  & \0870 &   7   &   0.7 &   yes \\
\hline
\end{tabular}
\end{center}

\caption{\label{tugg}The Tuggeranong house price data}
\end{table}

\subsection*{Suggested analysis}
Explore the data with \code{coplot()} using \code{Age} and \code{CentHeat} as
conditioning variables.

Choose a multiple regression model carefully and check for outliers, (that
is, for ``bargains'' and ``rip-offs'').


\clearpage\section{Yorke Penninsula wheat yield data}
\begin{tabbing}
\textbf{Category:} \= \kill
\textbf{Source:} \> K. W. Morris (private communication)\\
\textbf{Category:} \> Multiple regression.
\end{tabbing}

\subsection*{Description}
The annual yield of wheat in a marginal wheat growing district on the Yorke
Penninsula, South Australia, together with the rainfall for the three
growing months, for the years 1931--1955.  The year itself is potentially a
surrogate predictor to allow for improvements in varieties and farm
practice.  Yield is in bushels per acre, and rainfall is in inches.

\subsection*{Data}

\begin{table}[ht]
\begin{center}
\begin{tabular}{@{\protect\strutt}|crrrr|crrrr|}
\hline
Year & Rain0 & Rain1 & Rain2 & Yield & Year & Rain0 & Rain1 & Rain2 & Yield\\
\hline
  1931 &  .05 &  1.61 & 3.52 &   .31 & 1944 & 3.30 &  4.19 & 2.11 &  4.60\\
  1932 & 1.15 &   .60 & 3.46 &   .00 & 1945 &  .44 &  3.41 & 1.55 &   .35\\
  1933 & 2.22 &  4.94 & 3.06 &  5.47 & 1946 &  .50 &  3.26 & 1.20 &   .00\\
  1934 & 1.19 & 11.26 & 4.91 & 16.73 & 1947 &  .18 &  1.52 & 1.80 &   .00\\
  1935 & 1.40 & 10.95 & 4.23 & 10.54 & 1948 &  .80 &  3.25 & 3.55 &  2.98\\
  1936 & 2.96 &  4.96 &  .11 &  5.89 & 1949 & 7.08 &  5.93 &  .93 & 11.89\\
  1937 & 2.68 &   .67 & 2.17 &   .03 & 1950 & 2.54 &  4.71 & 2.51 &  6.56\\
  1938 & 3.66 &  8.49 & 11.95 & 16.03 & 1951 & 1.08 &  3.37 & 4.02 &  1.30\\
  1939 & 5.15 &  3.60 & 2.18 &  6.57 & 1952 &  .22 &  3.24 & 4.93 &   .03\\
  1940 & 6.44 &  2.69 & 1.37 &  8.43 & 1953 &  .55 &  1.78 & 1.97 &   .00\\
  1941 & 2.01 &  6.88 &  .92 &  8.68 & 1954 & 1.65 &  3.22 & 1.65 &  3.09\\
  1942 &  .73 &  3.30 & 3.97 &  2.49 & 1955 &  .72 &  3.42 & 3.31 &  2.72\\
  1943 & 2.52 &  1.93 & 1.16 &   .98 &      &      &       &       &\\
\hline
\end{tabular}
\end{center}

\caption{\label{sawheat} Yorke Penninsula wheat yield data}
\end{table}
The data is given in Table~\ref{sawheat} and may be read as a data frame from
file \code{sawheat.data}.

\subsection*{Suggested analysis}
Fit a multiple regression model and check the results, via residual plots
especially.


\clearpage\section{The Iowa wheat yield data}
\begin{tabbing}
\textbf{Category:} \= \kill
\textbf{Source:} \> CAED Report, 1964.  Quoted in Draper \& Smith.\\
\textbf{Category:} \>  Multiple regression; diagnostics.
\end{tabbing}

\subsection*{Description}
The data gives the pre-season and three growing months' precipitation, the
mean temperatures for the three growing months and harvest month, the year,
and the yield of wheat for the USA state of Iowa, for the years 1930--1962.

\subsection*{Data}

\begin{table}[ht]
\begin{center}
\small
\begin{tabular}{@{\protect\strutt}|ccccccccc|c|}
\hline
 Year & Rain0 & Temp1 & Rain1 &Temp2 & Rain2 &Temp3 & Rain3 &Temp4 &
 Yield\\
\hline
 1930 & 17.75 &  60.2 &  5.83 & 69.0 &  1.49 & 77.9 &  2.42 & 74.4 &  34.0\\
 1931 & 14.76 &  57.5 &  3.83 & 75.0 &  2.72 & 77.2 &  3.30 & 72.6 &  32.9\\
 1932 & 27.99 &  62.3 &  5.17 & 72.0 &  3.12 & 75.8 &  7.10 & 72.2 &  43.0\\
 1933 & 16.76 &  60.5 &  1.64 & 77.8 &  3.45 & 76.4 &  3.01 & 70.5 &  40.0\\
 1934 & 11.36 &  69.5 &  3.49 & 77.2 &  3.85 & 79.7 &  2.84 & 73.4 &  23.0\\
 1935 & 22.71 &  55.0 &  7.00 & 65.9 &  3.35 & 79.4 &  2.42 & 73.6 &  38.4\\
 1936 & 17.91 &  66.2 &  2.85 & 70.1 &  0.51 & 83.4 &  3.48 & 79.2 &  20.0\\
 1937 & 23.31 &  61.8 &  3.80 & 69.0 &  2.63 & 75.9 &  3.99 & 77.8 &  44.6\\
 1938 & 18.53 &  59.5 &  4.67 & 69.2 &  4.24 & 76.5 &  3.82 & 75.7 &  46.3\\
 1939 & 18.56 &  66.4 &  5.32 & 71.4 &  3.15 & 76.2 &  4.72 & 70.7 &  52.2\\
 1940 & 12.45 &  58.4 &  3.56 & 71.3 &  4.57 & 76.7 &  6.44 & 70.7 &  52.3\\
 1941 & 16.05 &  66.0 &  6.20 & 70.0 &  2.24 & 75.1 &  1.94 & 75.1 &  51.0\\
 1942 & 27.10 &  59.3 &  5.93 & 69.7 &  4.89 & 74.3 &  3.17 & 72.2 &  59.9\\
 1943 & 19.05 &  57.5 &  6.16 & 71.6 &  4.56 & 75.4 &  5.07 & 74.0 &  54.7\\
 1944 & 20.79 &  64.6 &  5.88 & 71.7 &  3.73 & 72.6 &  5.88 & 71.8 &  52.0\\
 1945 & 21.88 &  55.1 &  4.70 & 64.1 &  2.96 & 72.1 &  3.43 & 72.5 &  43.5\\
 1946 & 20.02 &  56.5 &  6.41 & 69.8 &  2.45 & 73.8 &  3.56 & 68.9 &  56.7\\
 1947 & 23.17 &  55.6 & 10.39 & 66.3 &  1.72 & 72.8 &  1.49 & 80.6 &  30.5\\
 1948 & 19.15 &  59.2 &  3.42 & 68.6 &  4.14 & 75.0 &  2.54 & 73.9 &  60.5\\
 1949 & 18.28 &  63.5 &  5.51 & 72.4 &  3.47 & 76.2 &  2.34 & 73.0 &  46.1\\
 1950 & 18.45 &  59.8 &  5.70 & 68.4 &  4.65 & 69.7 &  2.39 & 67.7 &  48.2\\
 1951 & 22.00 &  62.2 &  6.11 & 65.2 &  4.45 & 72.1 &  6.21 & 70.5 &  43.1\\
 1952 & 19.05 &  59.6 &  5.40 & 74.2 &  3.84 & 74.7 &  4.78 & 70.0 &  62.2\\
 1953 & 15.67 &  60.0 &  5.31 & 73.2 &  3.28 & 74.6 &  2.33 & 73.2 &  52.9\\
 1954 & 15.92 &  55.6 &  6.36 & 72.9 &  1.79 & 77.4 &  7.10 & 72.1 &  53.9\\
 1955 & 16.75 &  63.6 &  3.07 & 67.2 &  3.29 & 79.8 &  1.79 & 77.2 &  48.4\\
 1956 & 12.34 &  62.4 &  2.56 & 74.7 &  4.51 & 72.7 &  4.42 & 73.0 &  52.8\\
 1957 & 15.82 &  59.0 &  4.84 & 68.9 &  3.54 & 77.9 &  3.76 & 72.9 &  62.1\\
 1958 & 15.24 &  62.5 &  3.80 & 66.4 &  7.55 & 70.5 &  2.55 & 73.0 &  66.0\\
 1959 & 21.72 &  62.8 &  4.11 & 71.5 &  2.29 & 72.3 &  4.92 & 76.3 &  64.2\\
 1960 & 25.08 &  59.7 &  4.43 & 67.4 &  2.76 & 72.6 &  5.36 & 73.2 &  63.2\\
 1961 & 17.79 &  57.4 &  3.36 & 69.4 &  5.51 & 72.6 &  3.04 & 72.4 &  75.4\\
 1962 & 26.61 &  66.6 &  3.12 & 69.1 &  6.27 & 71.6 &  4.31 & 72.5 &  76.0\\
\hline
\end{tabular}
\end{center}
\caption{\label{iowheat} The Iowa historical wheat yield data}
\end{table}
The data is given in Table~\ref{iowheat} and may be read as a data frame from
file \code{iowheat.data}.

\subsection*{Suggested analysis}
Fit a multiple regression model and select carefully the predictors.  Work
either by backward elimination of forward selection.  Examine the residuals
by plotting them in turn against each predictor variable.

Consider the effect of adding quadratic terms in the predictors.

It is interesting to compare this set of data with the Yorke Penninsula
data for a similar period.


\clearpage\section{The gasoline yield data}
\begin{tabbing}
\textbf{Category:} \= \kill
\textbf{Source:} \> \textit{Estimate gasoline yields from crudes}\\
\> by Nilon H.~Prater, \textsf{Petroleum Refiner}, \textbf{35}, \#5.\\
\textbf{Category:} \> Analysis of variance, covariance, and multiple regression.\\
\>  Modern regression.
\end{tabbing}

\subsection*{Description}
The data gives the gasoline yield as a percent of crude oil, say $y$, and
four independent variables which may influence yield.  These are
\begin{description}

\item[$x_1$:] The crude oil gravity, in $^0$\textsf{API},

\item[$x_2$:] The crude oil vapour pressure,

\item[$x_3$:] The crude oil 10\% point, \textsf{ASTM},

\item[$x_4$:] The gasoline end point.

\end{description}

The data comes as 10 separate samples, and within each sample the values
for $x_1$, $x_2$, and $x_3$ are constant.

\subsection*{Data}
The data is shown in Table~\ref{oildata}, and is available as the file
\code{oil.data} in a form suitable for constructing a data frame.

\begin{table}[ht]
\begin{center}
\def\0{\hphantom{0}}
\begin{tabular}{@{\protect\strutt}|crrrrr|crrrrr|}
\hline
Sample& $x_1$&$x_2$&$x_3$&$x_4$&$y$ &Sample& $x_1$&$x_2$&$x_3$&$x_4$&$y$\\
\hline
 \01& 31.8& 0.2& 316& 365&  8.5 &\06& 40.0& 6.1& 217& 212&  7.4\\
 \01& 31.8& 0.2& 316& 379& 14.7 &\06& 40.0& 6.1& 217& 272& 18.2\\
 \01& 31.8& 0.2& 316& 428& 18.0 &\06& 40.0& 6.1& 217& 340& 30.4\\
%   &     &    &    &    &      &   &     &    &    &    &     \\
 \02& 32.2& 2.4& 284& 351& 14.0 &\07& 40.3& 4.8& 231& 307& 14.4\\
 \02& 32.2& 2.4& 284& 424& 23.2 &\07& 40.3& 4.8& 231& 367& 26.8\\
    &     &    &    &    &      &\07& 40.3& 4.8& 231& 395& 34.9\\
%   &     &    &    &    &      &   &     &    &    &    &     \\
 \03& 32.2& 5.2& 236& 267& 10.0 &\08& 40.8& 3.5& 210& 218&  8.0\\
 \03& 32.2& 5.2& 236& 360& 24.8 &\08& 40.8& 3.5& 210& 273& 13.1\\
 \03& 32.2& 5.2& 236& 402& 31.7 &\08& 40.8& 3.5& 210& 347& 26.6\\
%   &     &    &    &    &      &   &     &    &    &    &     \\
 \04& 38.1& 1.2& 274& 285&  5.0 &\09& 41.3& 1.8& 267& 235&  2.8\\
 \04& 38.1& 1.2& 274& 365& 17.6 &\09& 41.3& 1.8& 267& 275&  6.4\\
 \04& 38.1& 1.2& 274& 444& 32.1 &\09& 41.3& 1.8& 267& 358& 16.1\\
    &     &    &    &    &      &\09& 41.3& 1.8& 267& 416& 27.8\\
 \05& 38.4& 6.1& 220& 235&  6.9 & 10& 50.8& 8.6& 190& 205& 12.2\\
 \05& 38.4& 6.1& 220& 300& 15.2 & 10& 50.8& 8.6& 190& 275& 22.3\\
 \05& 38.4& 6.1& 220& 365& 26.0 & 10& 50.8& 8.6& 190& 345& 34.7\\
 \05& 38.4& 6.1& 220& 410& 33.6 & 10& 50.8& 8.6& 190& 407& 45.7\\
\hline
\end{tabular}
\end{center}

\caption{\label{oildata}The gasoline recovery data}
\end{table}

\subsection*{Suggested analysis}
Using \code{EndPt} as a covariate, check to see if differences between
samples can be accounted for by regression models on the other predictors.


More advanced:  Fit a two stratum ANOVA model using between and within
samples as the two strata.


\clearpage\section{The Michaelson and Morley speed of light data}
\begin{tabbing}
\textbf{Category:} \= \kill
\textbf{Source:} \> Weekes: A Genstat Primer.\\
\textbf{Category:} \> Analysis of Variance.
\end{tabbing}

\subsection*{Description}
The classical data of Michaelson and Morley on the speed of light.  The
data consists of five experiments, each consisting of 20 consecutive
``runs''.  The response is the speed of light measurement, suitably coded.
The data is here viewed as a randomized block experiment with
\textit{experiment} and \textit{run} as the factors.  \textit{run} may also be
considered a quantitative variate to account for linear (or polynomial)
changes in the measurement over the course of a single experiment.

\subsection*{Data}
The data is given in Table~\ref{morley} and may be read as a data frame
from file \code{morley.data} in a form suitable for constructing a data frame.

\begin{table}[ht]
\begin{center}
\begin{tabular}{|rrrrr|c@{\protect\strutt}|rrrrr|}
 \cline{1-5} \cline{7-11}
\multicolumn{5}{|c|}{Runs 1--10}&\hspace*{1cm}&\multicolumn{5}{c|}{Runs
11--20}\\
      ${\cal E}_1$ &    ${\cal E}_2$ &    ${\cal E}_3$ &    ${\cal E}_4$ &
      ${\cal E}_5$ &&    ${\cal E}_1$ &    ${\cal E}_2$ &    ${\cal E}_3$ &
      ${\cal E}_4$ &    ${\cal E}_5$\\
 \cline{1-5} \cline{7-11}
    850 &  960 &  880 &  890 &  890 && 1000 &  830 &  880 &  910 &  870\\
    740 &  940 &  880 &  810 &  840 && 980 &  790 &  910 &  920 &  870\\
    900 &  960 &  880 &  810 &  780 && 930 &  810 &  850 &  890 &  810\\
   1070 &  940 &  860 &  820 &  810 && 650 &  880 &  870 &  860 &  740\\
    930 &  880 &  720 &  800 &  760 && 760 &  880 &  840 &  880 &  810\\
    850 &  800 &  720 &  770 &  810 && 810 &  830 &  840 &  720 &  940\\
    950 &  850 &  620 &  760 &  790 &&1000 &  800 &  850 &  840 &  950\\
    980 &  880 &  860 &  740 &  810 &&1000 &  790 &  840 &  850 &  800\\
    980 &  900 &  970 &  750 &  820 && 960 &  760 &  840 &  850 &  810\\
    880 &  840 &  950 &  760 &  850 && 960 &  800 &  840 &  780 &  870\\
 \cline{1-5} \cline{7-11}
\end{tabular}
\end{center}

\caption{\label{morley}The Michaelson and Morley speed of light data}
\end{table}

\subsection*{Suggested analysis}
Using an single classification ANOVA model check for differences between
experiments and summarise your conclusions.

\clearpage\section{The rat genotype data}
\begin{tabbing}
\textbf{Category:} \= \kill
\textbf{Source:} \> Quoted in Scheffe, H.: \textit{The Analysis of Variance}\\
\textbf{Category:} \> Unbalanced double classification.
\end{tabbing}


\subsection*{Description}
Data from a foster feeding experiment with rat mothers and litters of four
different genotypes: $A$, $F$, $I$ and $J$.  The measurement is the litter
weight gain after a trial feeding period.

\subsection*{Data}

\begin{table}[ht]
\begin{center}
\begin{tabular}{@{\protect\strutt}|c|cccc|}
\hline
Litter's&\multicolumn{4}{c|}{Mother's Genotype}\\
Genotype &   $A$     &       $F$    &        $I$     &       $J$   \\
\hline
$A$     & 61.5    &     55.0   &      52.5    &     42.0  \\
        & 68.2    &     42.0   &      61.8    &     54.0  \\
        & 64.0    &     60.2   &      49.5    &     61.0  \\
        & 65.0    &            &      52.7    &     48.2  \\
        & 59.7    &            &              &     39.6  \\
\hline
$F$     & 60.3    &     50.8   &      56.5    &     51.3  \\
        & 51.7    &     64.7   &      59.0    &     40.5  \\
        & 49.3    &     61.7   &      47.2    &           \\
        & 48.0    &     64.0   &      53.0    &           \\
        &         &     62.0   &              &           \\
\hline
$I$     & 37.0    &     56.3   &      39.7    &     50.0  \\
        & 36.3    &     69.8   &      46.0    &     43.8  \\
        & 68.0    &     67.0   &      61.3    &     54.5  \\
        &         &            &              &     55.3  \\
        &         &            &              &     55.7  \\
\hline
$J$     & 59.0    &     59.5   &      45.2    &     44.8  \\
        & 57.4    &     52.8   &      57.0    &     51.5  \\
        & 54.0    &     56.0   &      61.4    &     53.0  \\
        & 47.0    &            &              &     42.0  \\
        &         &            &              &     54.0  \\
\hline
\end{tabular}
\end{center}
\caption{\label{genotype} The rat genotype data}
\end{table}
The data is given in Table~\ref{genotype} and may be read as a data frame from
file \code{genotype.data}.

\subsection*{Suggested analysis}
Fit a double classificaiton model.  Check for interaction using both a
formal analysis and graphically using \code{interaction.plot()}.  Test the
main effects and summarise.

\clearpage\section{Fisher's sugar beet data}
\begin{tabbing}
\textbf{Category:} \= \kill
\textbf{Source:} \> R. A. Fisher, \textsl{Design of Experiments}.\\
\textbf{Category:} \> Analysis of variance and covariance.
\end{tabbing}

\subsection*{Description}
A classical $3\times2^3$ randomized block experiment in four blocks of size
24.  The response is the total weight of sugarbeet roots off the plot, but
this is accompanied by the number of roots measured.  The suggestion is
that number of roots should be a covariate to allow for varying plot size.

The factors are \textsl{Variety}. (3 levels, $a$, $b$ and $c$), and N, P and K
each at 2 levels, present or absent.

\subsection*{Data}
The data is given in Table~\ref{sugar} and may be read from the file
\texttt{sugar.data} in a form suitable to construct a data frame.

\begin{table}[ht]
\begin{center}
\begin{tabular}{@{\protect\strutt}|cccc|*{4}{rr|}}
\hline
&&&&\multicolumn{2}{c|}{Block 1}&\multicolumn{2}{c|}{Block
2}&\multicolumn{2}{c|}{Block 3}&\multicolumn{2}{c|}{Block 4}\\
  V  &   N  &   P  &   K  &   No &  Wt  &   No &  Wt  &    No &  Wt  &   No
  &  Wt\\
\hline
 $a$ &  $-$ &  $-$ &  $-$ &  124 & 162  &  133 & 162  &   114 & 127  &  127 & 158\\
 $a$ &  $-$ &  $-$ &  $k$ &  131 & 152  &  161 & 164  &   130 & 141  &  145 & 188\\
 $a$ &  $-$ &  $p$ &  $-$ &  115 & 173  &  134 & 175  &   134 & 142  &  109 & 162\\
 $a$ &  $-$ &  $p$ &  $k$ &  126 & 140  &  133 & 158  &   106 & 148  &  132 & 160\\
 $a$ &  $n$ &  $-$ &  $-$ &  136 & 184  &  134 & 178  &   127 & 168  &  139 & 199\\
 $a$ &  $n$ &  $-$ &  $k$ &  134 & 112  &  156 & 193  &   101 & 171  &  138 & 191\\
 $a$ &  $n$ &  $p$ &  $-$ &  132 & 190  &  104 & 166  &   119 & 157  &  132 & 193\\
 $a$ &  $n$ &  $p$ &  $k$ &  120 & 175  &  147 & 155  &   107 & 139  &  148 & 192\\
 $b$ &  $-$ &  $-$ &  $-$ &  145 & 133  &  147 & 130  &   139 & 138  &  127 & 128\\
 $b$ &  $-$ &  $-$ &  $k$ &  156 & 117  &  152 & 137  &   107 & 121  &  147 & 147\\
 $b$ &  $-$ &  $p$ &  $-$ &  152 & 140  &  138 & 101  &   125 & 124  &  120 & 143\\
 $b$ &  $-$ &  $p$ &  $k$ &  137 & 127  &  145 & 132  &   125 & 132  &  143 & 139\\
 $b$ &  $n$ &  $-$ &  $-$ &  124 & 163  &  138 & 159  &   140 & 166  &  159 & 174\\
 $b$ &  $n$ &  $-$ &  $k$ &  136 & 143  &  142 & 144  &   133 & 142  &  148 & 159\\
 $b$ &  $n$ &  $p$ &  $-$ &  140 & 168  &  142 & 150  &   133 & 118  &  138 & 157\\
 $b$ &  $n$ &  $p$ &  $k$ &  146 & 144  &  135 & 160  &   138 & 155  &  140 & 153\\
 $c$ &  $-$ &  $-$ &  $-$ &  113 & 122  &  138 & 132  &   119 & 123  &  127 & 146\\
 $c$ &  $-$ &  $-$ &  $k$ &   91 & 107  &  149 & 171  &   118 & 142  &  129 & 151\\
 $c$ &  $-$ &  $p$ &  $-$ &  123 & 118  &  139 & 142  &   127 & 120  &  124 & 138\\
 $c$ &  $-$ &  $p$ &  $k$ &  129 & 140  &  126 & 115  &   129 & 130  &  142 & 152\\
 $c$ &  $n$ &  $-$ &  $-$ &  121 & 118  &  141 & 152  &   127 & 149  &  127 & 165\\
 $c$ &  $n$ &  $-$ &  $k$ &  126 & 148  &  128 & 152  &   107 & 147  &  110 & 136\\
 $c$ &  $n$ &  $p$ &  $-$ &  103 & 112  &  144 & 175  &   102 & 152  &  143 & 173\\
 $c$ &  $n$ &  $p$ &  $k$ &  120 & 162  &  125 & 160  &   129 & 173  &  137 & 185\\
\hline
\end{tabular}
\end{center}
\caption{\label{sugar}Fisher's sugar beet data}
\end{table}

\subsection*{Suggested analysis}
Analyse the data as a randomised block experiment with \code{Wt} as the
response and \code{No} as a covariate.  Prune the model of all unnecessary
interaction terms and summarise.


\clearpage\section{A barley split plot field trial}
\begin{tabbing}
\textbf{Category:} \= \kill
\textbf{Source:} \> Unknown.  Traditional data.\\
\textbf{Category:} \> Multistratum analysis of variance.
\end{tabbing}

\subsection*{Description}
An experiment involving barley varieties and manure (nitrogen) was conducted in 6
blocks of 3 whole plots.

Each whole plot was divided into 4 subplots.  Three varieties of barley were
used in the experiment with one variety being sown in each whole plot,
while the four levels of manure (0, 0.01, 0.02, and 0.04 tons per acre)
were used, one level in each of the four subplots of each whole plot.  In
the above table $V_i$ denotes the $i$th variety and $N_j$ denotes the $j$th
level of nitrogen.

\subsection*{Data}

\begin{table}[ht]
\begin{center}
\begin{tabular}{@{\protect\strutt}|c|c|rrrr||c|c|rrrr|}
\hline
Block & Variety & $N_1$ & $N_2$ & $N_3$ & $N_4$ &
Block & Variety & $N_1$ & $N_2$ & $N_3$ & $N_4$ \\
\hline
       & $V_1$ &   111  &  130  &  157  &  174  &
       & $V_1$ &    74  &   89  &   81  &  122  \\
  I    & $V_2$ &   117  &  114  &  161  &  141  &
 IV    & $V_2$ &    64  &  103  &  132  &  133  \\
       & $V_3$ &   105  &  140  &  118  &  156  &
       & $V_3$ &    70  &   89  &  104  &  117  \\
\hline
       & $V_1$ &    61  &   91  &   97  &  100  &
       & $V_1$ &    62  &   90  &  100  &  116  \\
 II    & $V_2$ &    70  &  108  &  126  &  149  &
  V    & $V_2$ &    80  &   82  &   94  &  126  \\
       & $V_3$ &    96  &  124  &  121  &  144  &
       & $V_3$ &    63  &   70  &  109  &   99  \\
\hline
       & $V_1$ &    68  &   64  &  112  &   86  &
       & $V_1$ &    53  &   74  &  118  &  113  \\
III    & $V_2$ &    60  &  102  &   89  &   96  &
 VI    & $V_2$ &    89  &   82  &   86  &  104  \\
       & $V_3$ &    89  &  129  &  132  &  124  &
       & $V_3$ &    97  &   99  &  119  &  121  \\
\hline
\end{tabular}
\caption{\label{barley} A split plot barley field trial}
\end{center}
\end{table}
The data is given in Table~\ref{barley} and may be read as a data frame from
file \code{barley.data}.

\subsection*{Suggested analysis}
Analyse as a split plot field experiment and summarise.


\clearpage\section{The snail mortality data}
\begin{tabbing}
\textbf{Category:} \= \kill
\textbf{Source:} \> Zoology Department, The University of Adelaide.\\
\textbf{Category:} \> Generalized Linear Modelling.
\end{tabbing}

\subsection*{Description}
Groups of 20 snails were held for periods of 1, 2, 3 or 4 weeks (exposure)
in carefully controlled conditions of temperature (3 levels) and relative
humidity (4 levels).  There were two species of snail, A and B, and the
experiment was designed as a $4\times3\times4\times2$ completely randomized
design.  At the end of the exposure time the snails were tested to see if
they had survived; this process itself is fatal for the animals.  The
object of the exercise was to model the probability of survival in terms of
the stimulus variables, and in particular to test for differences between
species.

The data is unusual in that in most cases fatalities during the experiment
were fairly small.

\subsection*{Data}

\begin{table}[ht]
\begin{center}
\begin{tabular}{@{\protect\strutt}|c|c||*{4}{rrr|}}
\hline
\multicolumn{2}{|c||}{}&\multicolumn{12}{c|}{Relative Humidity}\\
\multicolumn{2}{|c||}{}&\multicolumn{3}{c}{60.0\%}&\multicolumn{3}{c}{65.8\%}&%
  \multicolumn{3}{c}{70.5\%}&\multicolumn{3}{c|}{75.8\%}\\
\hline
\multicolumn{2}{|c||}{}&\multicolumn{3}{c|}{Temperature}&%
\multicolumn{3}{|c|}{Temperature}&\multicolumn{3}{c|}{Temperature}&%
\multicolumn{3}{c|}{Temperature}\\
\multicolumn{1}{|c}{Species}& Exposure &
             10 & 15 & 20 & 10 & 15 & 20 & 10 & 15 & 20 & 10 & 15 & 20\\
\hline
A   &  1 &    0 &  0 &  0 &  0 &  0 &  0 &  0 &  0 &  0 &  0 &  0 &  0\\
    &  2 &    0 &  1 &  1 &  0 &  1 &  0 &  0 &  0 &  0 &  0 &  0 &  0\\
    &  3 &    1 &  4 &  5 &  0 &  2 &  4 &  0 &  2 &  3 &  0 &  1 &  2\\
    &  4 &    7 &  7 &  7 &  4 &  4 &  7 &  3 &  3 &  5 &  2 &  3 &  3\\
\hline
\hline
B   &  1 &    0 &  0 &  0 &  0 &  0 &  0 &  0 &  0 &  0 &  0 &  0 &  0\\
    &  2 &    0 &  3 &  2 &  0 &  2 &  1 &  0 &  0 &  1 &  1 &  0 &  1\\
    &  3 &    7 & 11 & 11 &  4 &  5 &  9 &  2 &  4 &  6 &  2 &  3 &  5\\
    &  4 &   12 & 14 & 16 & 10 & 12 & 12 &  5 &  7 &  9 &  4 &  5 &  7\\
\hline
\end{tabular}
\end{center}

\caption{\label{snails} The snail mortality data}
\end{table}
The data is given in Table~\ref{snails} and may be read as a data frame from
file \code{snails.data}.

\subsection*{Suggested analysis}
The data is intersting in that although it has many extremely small cell
counts there is every indication that some of the likelihood ratio large
sample theory is quite safe.

Fit Binomial models to the data with either a logit or a probit link.  Show
that a model with parallel regressions on \code{Temp}, \code{Humid},
\texttt{Exposure} and \code{Exposure}$^2$ for each species (in the logit/probit
scale) is reasonable, and summarise.

\clearpage\section{The Kalythos blindness data}
\begin{tabbing}
\textbf{Category:} \= \kill
\textbf{Source:} \> S. D. Silvey: \textsl{Statistical Inference}.  (Fictitious?)\\
\textbf{Category:} \> Generalized linear modelling
\end{tabbing}

\subsection*{Description}
On the Greek island of Kalythos the male inhabitants suffer from a
congenital eye disease, the effects of which become more marked with
increasing age.  Samples of islander males of various ages  were tested for
blindness and the results recorded.

\subsection*{Data}

\begin{table}[ht]
\begin{center}
\begin{tabular}{@{\protect\strutt}|l|rrrrr|}
\hline
Age:        &  20 &  35 &  45 &  55 &  70 \\
\hline
No. tested: &  50 &  50 &  50 &  50 &  50 \\
\hline
No. blind:  &   6 &  17 &  26 &  37 &  44 \\
\hline
\end{tabular}
\end{center}

\caption{\label{kalythos}The Kalythos blindness data}
\end{table}
The data is given in Table~\ref{kalythos} and may be read as a data frame from
file \code{kalythos.data}.

\subsection*{Suggested analysis}
Using an logit or probit model estimate the LD50, that is, the age at which
the probability of blindness is $p=\frac{1}{2}$, together the standard
error.  Check how different the logit and probit models are in this respect.

\clearpage\section{The Stormer viscometer calibration data}
\begin{tabbing}
\textbf{Category:} \= \kill
\textbf{Source:} \> E. J. Williams: \textsl{Regression Analysis}, Wiley, 1959\\
\textbf{Category:} \> Nonlinear regression, special regression.
\end{tabbing}

\subsection*{Description}
The stormer viscometer measures the viscosity of a fluid by measuring the
time taken for an inner cylinder in the mechanism to perform a fixed number
of revolutions in response to an actuating weight.  The viscometer is
calibrated by measuring the time taken with varying weights while the
mechanism is suspended in fluids of accurately known viscosity.  The data
comes from such a calibration, and theoretical considerations suggest a
nonlinear relationship between time, weight and viscosity of the form
\[
T_i = \frac{\beta v_i}{w_i - \theta} + E_i
\]
where $\beta$ and $\theta$ are unknown parameters to be estimated.


\subsection*{Data}

\begin{table}[ht]
\begin{center}
\begin{tabular}{@{\protect\strutt}|r||r|r|r|}
\hline
&\multicolumn{3}{c|}{Weight}\\
Viscosity &   20   &    50   &   100\\
\hline
\hline
   14.7   &  35.6  &   17.6  &      \\
\hline
   27.5   &  54.3  &   24.3  &      \\
\hline
   42.0   &  75.6  &   31.4  &      \\
\hline
   75.7   & 121.2  &   47.2  &  24.6\\
\hline
   89.7   & 150.8  &   58.3  &  30.0\\
\hline
  146.6   & 229.0  &   85.6  &  41.7\\
\hline
  158.3   & 270.0  &  101.1  &  50.3\\
\hline
  161.1   &        &   92.2  &  45.1\\
\hline
  298.3   &        &  187.2  &  89.0\\
          &        &         &  86.5\\
\hline
\end{tabular}
\end{center}

\caption{\label{stormer}The Stormer viscometer calibration data}
\end{table}
The data is given in Table~\ref{stormer} and may be read as a data frame from
file \code{stormer.data}.

\subsection*{Suggested analysis}
Estimate the nonlinear regression model using the \code{nlr()} software.  A
suitable initial value may be obtained by writing the regression model in
the form
\[
w_iT_i = \beta v_i + \theta T_i + (w_i - \theta)E_i
\]
and regressing $w_iT_i$ on $v_i$ and $T_i$ using ordinary linear
regression.

\clearpage\section{The chlorine availability data}
\begin{tabbing}
\textbf{Category:} \= \kill
\textbf{Source:} \> Draper \& Smith, \textit{Applied Regression Analysis}, (adapted).\\
\textbf{Category:} \> Nonlinear regression
\end{tabbing}

\subsection*{Description}

The following set of industrial chemical data shows the amount of chlorine
available in a certain product at various times of testing after manufacture.
A nonlinear regression model for the chlorine decay of the form
\[
        Y = \beta_0 + \beta_1\exp(-\theta x)
\]
has been suggested on theoretical grounds, with $Y$ the amount remaining at
time $x$.

\subsection*{Data}

\begin{table}[ht]
\begin{center}
\begin{tabular}{@{\protect\strutt}|c|l||c|l||c|l|}
\hline
   Weeks  &  Percent available &   Weeks  &  Percent available &   Weeks  &
   Percent available \\
\hline
      8   &  49, 49&              20   &  42, 43, 42&      32   &  40, 41\\
     10   &  47, 47, 48, 48&      22   &  40, 41, 41&      34   &  40\\
     12   &  43, 45, 46, 46&      24   &  40, 40, 42&      36   &  38, 41\\
     14   &  43, 43, 45&          26   &  40, 41, 41&      38   &  40, 40\\
     16   &  43, 43, 44&          28   &  40, 41&          40   &  39\\
     18   &  45, 46&              30   &  38, 40, 40&      42   &  39\\
\hline
\end{tabular}
\end{center}

\caption{\label{chlorine} The chlorine availability data.}
\end{table}
The data is given in Table~\ref{chlorine} and may be read as a data frame from
file \code{chlorine.data}.

\subsection*{Suggested analysis}
Fit the nonlinear regression model using the \code{nlr()} function.  A
simple way to find an initial value is to guess a value for $\theta$, say
$\theta=1$ and plot $Y$ against $\exp(-\theta x)$.  Now repeatedly either
double or halve $\theta$ until the plot is near to linear.  (This can be
done very simply with the inbuilt line editor.)  Once an initial value for
$\theta$ is available, initial values for the others can be got by linear
regression.

Check the fitted model for suspicious data points.

\clearpage\section{The saturated steam pressure data}
\begin{tabbing}
\textbf{Category:} \= \kill
\textbf{Source:} \> Quoted in Draper \& Smith: \textit{Applied Regression Analysis\ldots}\\
\textbf{Category:} \> Nonlinear regression.
\end{tabbing}

\subsection*{Description}
The data gives the temperature ($^0$C) and pressure (Pascals) in a
saturated steam driven experimental device.  The relationship between
pressure, $Y$, and temperature, $x$, in saturated steam can be written as
\[
Y = \alpha \exp\left\{\beta x \over \gamma + x\right\} + E
\]
However a more realistic model may have the experimental errors
multiplicative rather than additive, in which case an analysis in the log
scale using the model
\[
\log Y = \log\alpha + \left\{\beta x \over \gamma + x\right\} + E
\]
may be more appropriate.

\subsection*{Data}

\begin{table}[ht]
\begin{center}
\begin{tabular}{@{\protect\strutt}|*{3}{rr|}}
\hline
  Temp  &   Press  &    Temp  &   Press  &  Temp  &   Press\\
\hline
     0  &    4.14  &      50  &   98.76  &    90  &  522.78\\
    10  &    8.52  &      60  &  151.13  &    95  &  674.32\\
    20  &   16.31  &      70  &  224.74  &   100  &  782.04\\
    30  &   32.18  &      80  &  341.35  &   105  &  920.01\\
    40  &   64.62  &      85  &  423.36  &        &\\
\hline
\end{tabular}
\end{center}
\caption{\label{steam} Temperature and pressure in saturated steam}
\end{table}
The data is given in Table~\ref{steam} and may be read as a data frame from
file \code{steam.data}.


\subsection*{Suggested analysis}
Fit both models and compare.  Initial values may be got by a similar method
to that employed for the Chlorine data, since again there is only one
nonlinear parameter.

\clearpage\section{Count Rumford's friction data}
\begin{tabbing}
\textbf{Category:} \= \kill
\textbf{Source:} \> Bates \& Watts: \textit{Nonlinear Regression Analysis\ldots}\\
\textbf{Category:} \> Nonlinear regression
\end{tabbing}

\subsection*{Description}
Data on the amount of heat generated by friction was obtained by Lord
Rumford in 1798.  A bore was fitted into a stationary cylinder and pressed
against the bottom by a screw.  The bore was turned by a team of horses
for 30 minutes, after which Lord Rumford ``suffered the thermometer to
remain in its place nearly three quarters of an hour, observing and noting
down, at small intervals of time, the temperature indicated by it''.

Newton's law of cooling suggests a nonlinear regression model of
the form
\[
Y = \beta_0 + \beta_1\exp(-\theta x)
\]
where $Y$ is the temperature and $x$ is the time in minutes.

\subsection*{Data}

\begin{table}[ht]
\begin{center}
\begin{tabular}{@{\protect\strutt}|*{2}{rr|}}
\hline
 Time  & Temp    &   Time &  Temp  \\
(min.) & ($^0$F) & (min.) & ($^0$F) \\
\hline
  4.0  &  126    &   24.0 & 115 \\
  5.0  &  125    &   28.0 & 114 \\
  7.0  &  123    &   31.0 & 113 \\
 12.0  &  120    &   34.0 & 112 \\
 14.0  &  119    &   37.5 & 111 \\
 16.0  &  118    &   41.0 & 110 \\
 20.0  &  116 & & \\
\hline
\end{tabular}
\end{center}
\caption{\label{rumford} The Rumford friction cooling data}
\end{table}
The data is given in Table~\ref{rumford} and may be read as a data frame from
file \code{rumford.data}.

\subsection*{Suggested analysis}
This data is mainly of historical interest.  Handle similarly to the
Chlorine data above.

\clearpage\section{The jellyfish data}
\begin{tabbing}
\textbf{Category:} \= \kill
\textbf{Source:} \> \textit{Interactive Statistics}, Ed.\ Don McNeil.\\
\textbf{Category:} \> Bivariate, two sample data.
\end{tabbing}

\subsection*{Description}
Two samples of jellyfish, from Danger Island and Salamander Bay
respectively, were measured for \code{length} and \code{width}.

\subsection*{Data}

\begin{table}[ht]
\begin{center}
\begin{tabular}{@{\protect\strutt}|*{4}{rr|}}
\hline
\multicolumn{4}{|c|}{Danger Island}&\multicolumn{4}{c|}
{Salamander Bay}\\
\hline
   Width &  Length & Width &  Length & Width &  Length & Width &  Length \\
\hline
  6.0 & 9.0 &  11.0 & 13.0 &  12.0 & 14.0 &  16.0 & 20.0\\
  6.5 & 8.0 &  11.0 & 14.0 &  13.0 & 17.0 &  16.0 & 20.0\\
  6.5 & 9.0 &  11.0 & 14.0 &  14.0 & 16.5 &  16.0 & 21.0\\
  7.0 & 9.0 &  12.0 & 13.0 &  14.0 & 19.0 &  16.5 & 19.0\\
  7.0 & 10.0 &  13.0 & 14.0 &  15.0 & 16.0 &  17.0 & 20.0\\
  7.0 & 11.0 &  14.0 & 16.0 &  15.0 & 17.0 &  18.0 & 19.0\\
  8.0 & 9.5 &  15.0 & 16.0 &  15.0 & 18.0 &  18.0 & 19.0\\
  8.0 & 10.0 &  15.0 & 16.0 &  15.0 & 18.0 &  18.0 & 20.0\\
  8.0 & 10.0 &  15.0 & 19.0 &  15.0 & 19.0 &  19.0 & 20.0\\
  8.0 & 11.0 &  16.0 & 16.0 &  15.0 & 21.0 &  19.0 & 22.0\\
  9.0 & 11.0 &  &        &  16.0 & 18.0 &  20.0 & 22.0\\
  10.0 & 13.0 & &        &  16.0 & 19.0 &  21.0 & 21.0\\
\hline
\end{tabular}
\end{center}
\caption{\label{jelly} The jellyfish data -- Danger Island and Salamander Bay}
\end{table}
The data is given in Table~\ref{jelly} and may be read as a data frame from
file \code{jellyfish.data}.

\subsection*{Suggested analysis}
Plot the two samples and mark in their convex hulls.  Test for differences
using Hotelling's $T^2$.  (A simple way of conducting the analysis is to
regress a dummy variable for \emph{Location} on \emph{Length} and
\emph{width} and to test the significance of both regression coefficients
simultaneously.

\clearpage\section{The Arch\ae{}ological pottery data}
\begin{tabbing}
\textbf{Category:} \= \kill
\textbf{Source:} \> Tubb, A. \emph{et al.} Arch\ae{}ometry, \textbf{22}, 153--171, (1980)\\
\textbf{Category:} \>  Multivariate analysis
\end{tabbing}

\subsection*{Description}
    The data arises from a chemical analysis of 26 samples of pottery found
    at Romano-British kiln sites in Wales, Gwent and the New Forest.  The
    variables describe the composition, in terms of various metals, and are
    expressed as percentages of the oxides of the metals.

    The metals are aluminium, iron, magnesium, calcium and sodium and the
    sites are
\begin{center}
L: Llanederyn, \quad C: Caldicot\quad  I: Island Thorns\quad A: Ashley Rails
\end{center}

\subsection*{Data}

\begin{table}[ht]
\begin{center}
\begin{tabular}{@{\protect\strutt}|*{2}{c|rrrrr|}}
\hline
 Site & Al & Fe & Mg & Ca & Na & Site & Al & Fe & Mg & Ca & Na\\
\hline
 L & 14.4 & 7.00 & 4.30 & 0.15 & 0.51 & C & 11.8 & 5.44 & 3.94 & 0.30 & 0.04\\
 L & 13.8 & 7.08 & 3.43 & 0.12 & 0.17 & C & 11.6 & 5.39 & 3.77 & 0.29 & 0.06\\
 L & 14.6 & 7.09 & 3.88 & 0.13 & 0.20 & I & 18.3 & 1.28 & 0.67 & 0.03 & 0.03\\
 L & 11.5 & 6.37 & 5.64 & 0.16 & 0.14 & I & 15.8 & 2.39 & 0.63 & 0.01 & 0.04\\
 L & 13.8 & 7.06 & 5.34 & 0.20 & 0.20 & I & 18.0 & 1.50 & 0.67 & 0.01 & 0.06\\
 L & 10.9 & 6.26 & 3.47 & 0.17 & 0.22 & I & 18.0 & 1.88 & 0.68 & 0.01 & 0.04\\
 L & 10.1 & 4.26 & 4.26 & 0.20 & 0.18 & I & 20.8 & 1.51 & 0.72 & 0.07 & 0.10\\
 L & 11.6 & 5.78 & 5.91 & 0.18 & 0.16 & A & 17.7 & 1.12 & 0.56 & 0.06 & 0.06\\
 L & 11.1 & 5.49 & 4.52 & 0.29 & 0.30 & A & 18.3 & 1.14 & 0.67 & 0.06 & 0.05\\
 L & 13.4 & 6.92 & 7.23 & 0.28 & 0.20 & A & 16.7 & 0.92 & 0.53 & 0.01 & 0.05\\
 L & 12.4 & 6.13 & 5.69 & 0.22 & 0.54 & A & 14.8 & 2.74 & 0.67 & 0.03 & 0.05\\
 L & 13.1 & 6.64 & 5.51 & 0.31 & 0.24 & A & 19.1 & 1.64 & 0.60 & 0.10 & 0.03\\
 L & 12.7 & 6.69 & 4.45 & 0.20 & 0.22&&&&&&\\
         L  &     12.5  &     6.44  &    3.94 &     0.22 &   0.23&&&&&&\\
\hline
\end{tabular}
\end{center}
\caption{\label{pottery} The pottery composition data}
\end{table}
The data is given in Table~\ref{pottery} and may be read as a data frame from
file \code{pottery.data}.

\subsection*{Suggested analysis}
Investigate both numerically and graphically using a simple discriminant
analysis.  Exhibit the four samples using the first two discriminant
functions as coordinate axes.  Summarise.


\clearpage\section{The Beaujolais quality data}
\begin{tabbing}
\textbf{Category:} \= \kill
\textbf{Source:} \> Quoted in Weekes: \emph{A Genstat Primer}\\
\textbf{Category:} \> Multivariate analsysis
\end{tabbing}

\subsection*{Description}
Quality measurements for some identified samples of young Beaujolais.
Extracted from Table 1 in M.~G.~Jackson, \emph{et al}: Red wine quality:
correlations between colour, aroma and flavour and pigment and other
parameters of young Beaujolais, \emph{Journal of Science of Food and
Agriculture}, \textbf{29}, 715--727, (1978).

\subsection*{Data}

\begin{table}[ht]
\begin{center}
\begin{tabular}{@{\protect\strutt}|*{2}{c|rrrr|}}
\hline
Label & OQ & AC & pH & TSO & Label & OQ & AC & pH & TSO \\
\hline
 A & 13.54 & 1.51 & 3.36 & 13.8 & I &  12.25 & 1.32 & 3.38 & 1.4 \\
 B & 12.58 & 1.35 & 3.15 & 5.2  & J &  14.04 & 1.52 & 3.61 & 4.5 \\
 C & 11.83 & 1.09 & 3.30 & 10.6 & K &  12.67 & 1.62 & 3.38 & 0.4 \\
 D & 12.83 & 1.15 & 3.41 & 2.2  & L &  13.54 & 1.57 & 3.55 & 7.9 \\
 E & 12.83 & 1.32 & 3.44 & 2.3  & M &  13.75 & 1.63 & 3.34 & 6.3 \\
 F & 12.12 & 1.23 & 3.31 & 10.5 & N &  9.63 & 0.78 & 3.19 &40.4 \\
 G & 11.29 & 1.14 & 3.49 & 2.5  & O &  12.42 & 1.14 & 3.31 & 3.1 \\
 H & 12.79 & 1.22 & 3.56 & 16.7 &   &   & & & \\
\hline
\end{tabular}
\end{center}
\caption{\label{wine} Quality measurements on young Beaujolais wine samples}
\end{table}
The data is given in Table~\ref{wine} and may be read as a data frame from
file \code{beaujolais.data}.

\subsection*{Suggested analysis}
Look at ways of exhibiting the data graphically.  Consider a principal
component analysis using the correlation matrix and look for any wild
outliers.

\clearpage\section{The painters data of de Piles}
\begin{tabbing}
\textbf{Category:} \= \kill
\textbf{Source:} \>  Weekes: A Genstat Primer.\\
\textbf{Category:} \> Multivariate Analysis: Discriminant Analysis.
\end{tabbing}

\subsection*{Description}
The data shows the subjective assessment, on a 0--20 integer scale, of 54
classical painters.  The painters were assessed on four characteristics:
composition, drawing, colour and expression.  The data is due to the
Eighteenth century art critic, de Piles.

The School to which a painter belongs is indicated by a letter code as follows:
\begin{center}
{\newdimen\wid
\setbox0=\hbox{Seventeenth Century\quad}
\wid=\wd0
\begin{tabular}{@{\protect\strutt}|*{2}{cp{\wid}|}}
\hline
A & Renaissance & E & Lombard          \\
B & Mannerist   & F & Sixteenth Century\\
C & Seicento    & G & Seventeenth Century \\
D & Venetian    & H & French              \\
\hline
\end{tabular}
}
\end{center}
%
% Number of observations: 54
% Number of variables: 5 (including the school)
% Source: Davenport and Studdert-Kennedy, op.cit. "

\subsection*{Data}
{\def\0{\phantom{0}}
\begin{table}[ht]
\begin{center}
\small
\begin{tabular}{|l|cccc|c|}
\hline
                & Composition &  Drawing &  Colour  & Expression &  School\\
\hline
Da Udine        &     10      & \08    &  16    &  \03     &  A\\
Da Vinci        &     15      &  16    & \04    &   14     &  A\\
Del Piombo      &    \08      &  13    &  16    &  \07     &  A\\
Del Sarto       &     12      &  16    & \09    &  \08     &  A\\
Fr.\ Penni      &    \00      &  15    & \08    &  \00     &  A\\
Guilio Romano   &     15      &  16    & \04    &   14     &  A\\
Michelangelo    &    \08      &  17    & \04    &  \08     &  A\\
Perino del Vaga &     15      &  16    & \07    &  \06     &  A\\
Perugino        &    \04      &  12    &  10    &  \04     &  A\\
Raphael         &     17      &  18    &  12    &   18     &  A\\
\hline
F. Zucarro      &     10      &  13    & \08    &  \08     &  B\\
Fr.\ Salviata   &     13      &  15    & \08    &  \08     &  B\\
Parmigiano      &     10      &  15    & \06    &  \06     &  B\\
Primaticcio     &     15      &  14    & \07    &   10     &  B\\
T. Zucarro      &     13      &  14    &  10    &  \09     &  B\\
Volterra        &     12      &  15    & \05    &  \08     &  B\\
\hline
Barocci         &     14      &  15    & \06    &   10     &  C\\
Cortona         &     16      &  14    &  12    &  \06     &  C\\
Josepin         &     10      &  10    & \06    &  \02     &  C\\
L. Jordaens     &     13      &  12    & \09    &  \06     &  C\\
Testa           &     11      &  15    & \00    &  \06     &  C\\
Vanius          &     15      &  15    &  12    &   13     &  C\\
\hline
Bassano         &    \06      & \08    &  17    &  \00     &  D\\
Bellini         &    \04      & \06    &  14    &  \00     &  D\\
Giorgione       &    \08      & \09    &  18    &  \04     &  D\\
Murillo         &    \06      & \08    &  15    &  \04     &  D\\
Palma Giovane   &     12      & \09    &  14    &  \06     &  D\\
Palma Vecchio   &    \05      & \06    &  16    &  \00     &  D\\
Pordenone       &    \08      &  14    &  17    &  \05     &  D\\
Tintoretto      &     15      &  14    &  16    &  \04     &  D\\
Titian          &     12      &  15    &  18    &  \06     &  D\\
Veronese        &     15      &  10    &  16    &  \03     &  D\\
\hline
Albani          &     14      &  14    &  10    &  \06     &  E\\
Caravaggio      &    \06      & \06    &  16    &  \00     &  E\\
Corregio        &     13      &  13    &  15    &   12     &  E\\
Domenichino     &     15      &  17    & \09    &   17     &  E\\
Guercino        &     18      &  10    &  10    &  \04     &  E\\
Lanfranco       &     14      &  13    &  10    &  \05     &  E\\
The Carraci     &     15      &  17    &  13    &   13     &  E\\
\hline
Durer           &    \08      &  10    &  10    &  \08     &  F\\
Holbein         &    \09      &  10    &  16    &   13     &  F\\
Pourbus         &    \04      &  15    & \06    &  \06     &  F\\
Van Leyden      &    \08      & \06    & \06    &  \04     &  F\\
\hline
Diepenbeck      &     11      &  10    &  14    &  \06     &  G\\
J. Jordaens     &     10      & \08    &  16    &  \06     &  G\\
Otho Venius     &     13      &  14    &  10    &   10     &  G\\
Rembrandt       &     15      & \06    &  17    &   12     &  G\\
Rubens          &     18      &  13    &  17    &   17     &  G\\
Teniers         &     15      &  12    &  13    &  \06     &  G\\
Van Dyck        &     15      &  10    &  17    &   13     &  G\\
\hline
Bourdon         &     10      & \08    & \08    &  \04     &  H\\
Le Brun         &     16      &  16    & \08    &   16     &  H\\
Le Suer         &     15      &  15    & \04    &   15     &  H\\
Poussin         &     15      &  17    & \06    &   15     &  H\\
\hline
\end{tabular}
\end{center}
\caption{\label{paint} The subjective assessment data of de Piles}
\end{table}
}
The data is given in Table~\ref{paint} and may be read as a data frame from
file \code{painters.data}.

\subsection*{Suggested analysis}
Using a multivariate analysis of variance check for differences between
schools.  Use the likelihood ratio test.  Also find the canonical
$F-$statistics and discriminant functions.

Plot the painters on the first two discriminant function axes and use the
school symbol as a plotting character.  Mark in the convex hulls of the
schools.  Using \code{identify()} find interactively some of the painers
that appear to lie towards the extremes of the plot, or who deviate
considerably from their school mean.

\end{document}

